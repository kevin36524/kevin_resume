% resume.tex
%
% (c) 2002 Matthew Boedicker <mboedick@mboedick.org> (original author) http://mboedick.org
% (c) 2003 David J. Grant <dgrant@ieee.org> http://www.davidgrant.ca
% (c) 2007 Todd C. Miller <Todd.Miller@courtesan.com> http://www.courtesan.com/todd
%
%This work is licensed under the Creative Commons Attribution-NonCommercial-ShareAlike License. To view a copy of this license, visit http://creativecommons.org/licenses/by-nc-sa/1.0/ or send a letter to Creative Commons, 559 Nathan Abbott Way, Stanford, California 94305, USA.

\documentclass[a4paper,11pt,times]{res}

%-----------------------------------------------------------
\usepackage{textcomp}
\usepackage{hyperref}
\usepackage{fullpage}
\usepackage{graphicx}
\usepackage{verbatim}
\usepackage{longtable}
%\usepackage[empty]{fullpage}
\usepackage{color}
\usepackage{pgfbaseimage}
%\input pgfbaseimage.tex
%\input pgfbaseimage.tex
\definecolor{mygrey}{gray}{0.81}
\definecolor{orange}{rgb}{1,0.5,0}
\definecolor{blue}{rgb}{0,0,1}
%\raggedbottom
%\raggedright
% Adjust margins to 12.5mm on all sides
\addtolength{\oddsidemargin}{-27mm}
\addtolength{\evensidemargin}{-27mm}
\addtolength{\textwidth}{25mm}
\addtolength{\topmargin}{-12.5mm}
\addtolength{\textheight}{25mm}

%-----------------------------------------------------------
%kevin commands
\pgfdeclareimage[height=0.6cm]{yahoologo}{yahoo.jpeg}
%\usepackage[absolute,overlay]{textpos}
%\setlength{\TPHorizModule}{1mm}
%\setlength{\TPVertModule}{1mm}
\newcommand{\YahooLogo}{
\begin{textblock}
  \pgfuseimage{yahoologo}
\end{textblock}
}
%\pgfdeclareimage[height=18.7mm,width=33.3mm]{yahoo}{yahoo}
  
%%%%%
%Custom commands
\newcommand{\resitem}[1]{\item #1 \vspace{-2pt}}
\newcommand{\smalitem}[1]{\item #1 \vspace{-4pt}}
\newcommand{\resheading}[1]{{\large \colorbox{mygrey}{\begin{minipage}{\textwidth}{\textbf{#1 \vphantom{p\^{E}}}}\end{minipage}}\vspace{4pt}}}
\newcommand{\ressubheading}[4]{
\begin{tabular*}{172mm}{l@{\extracolsep{\fill}}r}
		\textbf{#1} & #2 \\
		\textit{#3} & \textit{#4} \\
\end{tabular*}\vspace{4pt}}
\newcommand{\mysubheading}[2]{
\begin{tabular*}{172mm}{l@{\extracolsep{\fill}}r}
		\textbf{#1} & \textit{#2} \\
\end{tabular*}\vspace{-1pt}}
%-----------------------------------------------------------

\begin{document}

\vspace{-2pt}
\begin{tabular*}{160mm}{l}
\textbf{Kevin Patel} \\
Director, Software Apps Engineering at Yahoo \\
\textbf{Email:} patelkev@yahooinc.com \\
\textbf{Mobile:} +1 408-594-6820 \\
\end{tabular*}

\resheading{Work Experience}
\begin{itemize}

\item
\mysubheading{Director of Engineering for Yahoo Mail(iOS) }{(July 2022 onwards)}
\begin{itemize}
\vspace{-2pt}
\smalitem{\textbf{Working with the team to develop and release a newer version of Yahoo Mail iOS App -} Managed a team of talented iOS developers to deliver new Mail App with utmost stability and using the latest technological advances like Reactive programming, Combine and SwiftUI.}
\smalitem{\textbf{Expansion of Yahoo Mail iOS team -} Worked with strategic business and HR partners to come up with the expansion plan for expanding our Yahoo Mail iOS team to offshore location. This includes building of a new team in Bangalore, India and also get more resources by partnering with external contractors. }
\end{itemize}

\vspace{2pt}

\item
\mysubheading{Engineering Manager for Yahoo Mail(iOS) }{(May 2019 - July 2022)}
\begin{itemize}
\vspace{-2pt}
\smalitem{\textbf{Launched new version of Yahoo Mail iOS App -} Managed a team of talented iOS developers to deliver new Mail App with utmost stability and with only 2 MB increase in the app size while serving both the old version and the new version during the time of phased release. As a part of this revamp and other growth and stability efforts we were able to grow our App DAUs from 10 million to 14 million.}
\smalitem{\textbf{Developed Framework for hosting other properties into Yahoo Mail App and other iOS Apps -} I helped in creating an architecture for other non-mail Yahoo properties to create an experience which can be hosted within a tab in Yahoo Mail. The task was quite challenging as we wanted the other properties to develop experiences which can be hosted in other Yahoo Apps too without having to re-code. I made significant code contributions to the project. This architecture came out to be very impressive and enabled the external non-Mail team to rapidly develop a reusable tab experience without increasing the compile time of Yahoo Mail App and significantly reducing their development and compile times.}
\smalitem{\textbf{Helped other apps to leverage our Yahoo Mail Code -} Me and my team helped other Verizon Media teams like Yahoo App and AOL mail team to use our battle tested Yahoo Mail code, this enabled our teams to unify our code bases and enabled us to leverage extra developers, while ensuring that our users get the best and most stable app experience with a crash rate of less than 0.01\%. }
\smalitem{\textbf{Instilled optimal resource management -} Came up with proper resource management and splitted the standup into smaller focus groups to ensure that the developers are focused and the product is delivered on time while ensuring that the developers are not overloaded.}
\smalitem{\textbf{Improved the team morale -} I understood the painpoints my team shared and introduced the program of giving 1 week to each developer per quarter to learn new skills and explore other areas. This facilitated many developers to learn Machine Learning, Swift UI, Realm DB, Firebase, etc.}
\smalitem{\textbf{Facilitated the development of various features for new Yahoo Mail -} Worked with PMs, developers and architects of various teams to ensure we get optimal timely solutions for mulitple features like:
\begin{itemize}
\smalitem {iOS 13 widgets on SwiftUI.}
\smalitem {A fully flushed out iPad experience with keyboard shortcuts which can also work as a Mac app for Apple silicon Macs.}
\smalitem {NFL sports tab experience.}
\smalitem {Coronavirus and election notifications.}
\smalitem {Font scaling widget without horizontal scrolling for HTML based emails.}
\smalitem {Auto reply suggestions based on ML and hosted on AWS kubernetes autoscaling cluster.}
\end{itemize}
 }

\end{itemize}

\vspace{2pt}

\item
\mysubheading{iOS Mobile Developer for Yahoo Mail }{(Sept 2015 - May 2019)}
\begin{itemize}
\vspace{-2pt}
\smalitem{\textbf{Worked on launching initial version of Yahoo Mail iOS App -} Worked on many small and big features on getting the very first version of Yahoo Mail iOS app.}
\smalitem{\textbf{IMAP-In support for Yahoo Mail iOS -} Implemented all the parts necessary to support multiple Accounts like GMail, Outlook and AOL into Yahoo Mail iOS.}
\smalitem{\textbf{Dynamic update Pipeline -} Implemented the dynamic-update pipeline for updating the javascript resources which allowed us to roll out new features for Thanks Giving and Holiday stationery.}
\smalitem{\textbf{API migration -} Updated to the latest search and backend APIs with phased rollout and backward compatibility.}
\smalitem{\textbf{New features for iOS mobile Mail -} Implemented features like Flight Cards, Stationery, Document Preview and iOS-10 actionable push notifications.}

\end{itemize}

\vspace{10pt}

\item
\mysubheading{Web FrontEnd Developer for Yahoo }{(Aug 2010 - Sept 2015)}
\begin{itemize}
\vspace{-2pt}
\smalitem{\textbf{Stationery with Paperless Post in Yahoo Mail -} Implemented stationery which enabled sending beautiful emails on Yahoo Mail Desktop. Also devised the system for Paperless Post to share the new email Cards to Yahoo Mail.}
\smalitem{\textbf{Document Preview in Yahoo Mail -} Implemented highly optimal and performant HTML based Document preview on Yahoo Mail Frontend.}
\smalitem{\textbf{Minty Fresh Calendar and Notepad -} Came up with an architecture by which we can have both the old and the new versions of calendar working in tandem. The solution that I designed, not only prevented code duplication but also provided complete code isolation and no impact in performance.}
\smalitem{\textbf{Corp Calendar -} Implemented the key front-end pieces in Corp Calendar, like free busy module, room availability and location autocomplete.}
\end{itemize}

\end{itemize}

\resheading{Patents and Important Awards}
\begin{itemize}
\smalitem{\textbf{CEO Challenge winner (2014-2015)  -} As a captain of the CEO Challenge team came up with the idea and formed a team to help get us extra revenue. The idea was approved by our CEO and we were able to develop and deploy the idea which ultimately got approx 10.9 million USD incremental gross revenue per year for Yahoo.}
\smalitem{\textbf{Patent ID15-11332 - July 2015} ZIP/Rar/Compressed folder and its documents preview inside an email client.}
\smalitem{\textbf{Patent ID15-11161 - May 2015} Occasion/Purpose based personalized automated email composition using machine learning.}
\smalitem{\textbf{Patent ID10-7100 - Nov 2010} Monetizing UGC by embeding ads.}
\end{itemize}


\resheading{Technical Skills}
\begin{itemize}
\smalitem{Objective C, Swift, HTML, CSS, JavaScript, NodeJS}
\end{itemize} 

\resheading{Academics}
\begin{tabular*}{105mm}{l l l l}
\hline
\textbf{Institution}&\textbf{Degree/Major}&\textbf{Year}&\textbf{CPI/\%}\\
\hline
IIT Bombay & MTech - Microelectronics & 2010 & 9.34\\
Gujarat Univ. & BE- Elect. \& Comm. & 2007 & 72.75\\
Gujarat Board & Intermediate/ +2 & 2003 & 67.54\\
Gujarat Boad & Matriculation & 2001 & 84.57\\
\hline
\end{tabular*}

\end{document}
